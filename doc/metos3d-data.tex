%
% Metos3D: A Marine Ecosystem Toolkit for Optimization and Simulation in 3-D
% Copyright (C) 2012  Jaroslaw Piwonski, CAU, jpi@informatik.uni-kiel.de
%
% This program is free software: you can redistribute it and/or modify
% it under the terms of the GNU General Public License as published by
% the Free Software Foundation, either version 3 of the License, or
% (at your option) any later version.
%
% This program is distributed in the hope that it will be useful,
% but WITHOUT ANY WARRANTY; without even the implied warranty of
% MERCHANTABILITY or FITNESS FOR A PARTICULAR PURPOSE.  See the
% GNU General Public License for more details.
%
% You should have received a copy of the GNU General Public License
% along with this program.  If not, see <http://www.gnu.org/licenses/>.
%

%
%	document class
%
\documentclass{article}
%
%	packages
%
\usepackage{natbib}
\usepackage{amsmath}
\usepackage[pdfborder={0 0 0},colorlinks,urlcolor=blue]{hyperref}
%
%	commands
%
\parindent0em
%
%	BEGIN DOCUMENT
%
\begin{document}
%
%	title
%
\title{
Metos3D \\
\medskip
\texttt{data}
}
\author{
Jaroslaw Piwonski\thanks{\texttt{jpi@informatik.uni-kiel.de}} \,,
Thomas Slawig\thanks{\texttt{ts@informatik.uni-kiel.de},
both: Department of Computer Science, Algorithmic Optimal Control -- Computational Marine Science,
Excellence Cluster The Future Ocean, Christian-Albrechts-Platz 4, 24118 Kiel, Germany.}
}
\date{\today}
\maketitle

%
%	Introdution
%
\section{Introdution}

Metos3D was developed to use data provided by the
\href{http://www.ldeo.columbia.edu/~spk/Research/TMM/tmm.html}{Transport Matrix Method}
in the first place \citep[cf. ][]{KhViCa05}.
%
This original data is maintained by
\href{http://www.ldeo.columbia.edu/~spk/}{Samar Khatiwala}.
%
Metos3D provides already prepared data for usage.

%
%	Data
%
\section{Data}

Currently a $ 2.8125^\circ $ ($ 2.8^\circ $ for short) spatial resolution is provided.
The vector length, the profile count and the maximum number of layers is $ 52749 $, $ 4448 $
and $ 15 $, respectively. The temporal resolution is $ 3 $ hours,
assuming $ 360 $ days and $ 2880 $ steps per year.

%
%	Data Hierarchy
%
\subsection{Data Hierarchy}

The data is organized in a hierarchy of directories starting with the
root directory named \texttt{TMM}.
The \textbf{main distinction} between data files is the \textbf{spatial resolution}:
\begin{verbatim}
.../TMM/
    2.8/
\end{verbatim}

For a spatial resolution you will find a distinction
into \textbf{geometry}, \textbf{initialization}, \textbf{forcing} and
\textbf{transport} within that directory:
\begin{verbatim}
.../TMM/2.8/
    Geometry/
    Initialization/
    Forcing/
    Transport/
\end{verbatim}

The geometry and initialization directories have no further subdirectories.
The forcing is divided into \textbf{boundary} and \textbf{domain conditions}.
\begin{verbatim}
.../TMM/2.8/Forcing/
    BoundaryCondition/
    DomainCondition/
\end{verbatim}

The transport matrices are organized within the transport directory.
The first level of subdirectories denote the \textbf{kind} of matrices (see \cite{KhViCa05}).
The second distinguishes between \textbf{temporal resolutions}.
\begin{verbatim}
.../TMM/2.8/Transport/
    Matrix5_4/
        1dt/
\end{verbatim}

%
%	References	%%%%%%%%%%%%%%%%%%%%%%%%%%%%%%%%%%%%%%%%%%%%%%%%
%
\bibliographystyle{plain}
%\bibliography{/Users/jpicau/Documents/ARBEIT/CODE/Literature/literature}
\begin{thebibliography}{1}

\bibitem{KhViCa05}
S.~Khatiwala, M.~Visbeck, and M.A. Cane.
\newblock Accelerated simulation of passive tracers in ocean circulation
  models.
\newblock {\em Ocean Modelling}, 9(1):51--69, 2005.

\end{thebibliography}


\end{document}
