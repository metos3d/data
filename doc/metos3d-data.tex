%
%   metos3d-data.tex
%
%   Author:
%       Jaroslaw Piwonski, CAU Kiel, jpi@informatik.uni-kiel.de
%

%
%	document class
%
\documentclass{article}
%
%	packages
%
\usepackage{amsmath}
\usepackage[pdfborder={0 0 0},colorlinks,urlcolor=blue]{hyperref}
%
%	commands
%
\parindent0em
%
%	new commands
%
\newcommand{\VERSION}{{0.1}}
\newcommand{\Metos}{{\sc Metos3D}}
%
%	BEGIN DOCUMENT
%
\begin{document}
%
%	title
%
\title{
\Metos \\
\bigskip
Data \\
{\small Version \VERSION}
}
\author{
Jaroslaw Piwonski\thanks{\texttt{jpi@informatik.uni-kiel.de}} \,,
Thomas Slawig\thanks{\texttt{ts@informatik.uni-kiel.de},
both: Department of Computer Science, Algorithmic Optimal Control -- Computational Marine Science,
Excellence Cluster The Future Ocean, Christian-Albrechts-Platz 4, 24118 Kiel, Germany.}
}
\date{\today}
\maketitle

%
%	Introdution
%
\section{Introdution}

\Metos\ was developed to use data provided by the
\href{http://www.ldeo.columbia.edu/~spk/Research/TMM/tmm.html}{Transport Matrix Method}
in the first place (see \cite{KhViCa05}).
This data is maintained by
\href{http://www.ldeo.columbia.edu/~spk/}{Samar Khatiwala}.
\Metos\ supplies you with a
MATLAB\footnote{MATLAB is a registered trademark of The MathWorks, Inc., 
\texttt{www.mathworks.com}}
script to prepare the data for usage.

%
%	Quick Start
%
\section{Quick Start}

Assuming you are in a data directory do the following:

\begin{enumerate}
\item Download the required data.
\begin{verbatim}
DATA $>
wget http://www.ldeo.columbia.edu/%7Espk/Research/TMM/MIT_Matrix_Global_2.8deg.tar.gz  # 503M !!! 
wget http://www.ldeo.columbia.edu/%7Espk/Code/Matlab/Matrix.tar.gz                     #   9K
wget http://www.ldeo.columbia.edu/%7Espk/Code/Matlab/Misc.tar.gz                       #  23K
wget http://www.ldeo.columbia.edu/%7Espk/Code/Matlab/MITgcm.tar.gz                     #  16K
wget http://www.ldeo.columbia.edu/%7Espk/Code/Matlab/PETSC.tar.gz                      #  14K
wget http://www.informatik.uni-kiel.de/~algopt/metos3d/data/Metos3DData.tar.gz         #   3K
wget http://www.informatik.uni-kiel.de/~algopt/metos3d/data/prepareMetos3DData.m       #   2K
\end{verbatim}
\item Start MATLAB and call
\begin{verbatim}
>>
prepareMetos3DData
\end{verbatim}
\end{enumerate}


%
%	Data
%
\section{Data}

Currently a $ 2.8125^\circ $ ($ 2.8^\circ $ for short) spatial resolution is provided.
The vector length, the profile count and the maximum number of layers is $ 52749 $, $ 4448 $
and $ 15 $, respectively. The temporal resolution is $ 3 $ hours,
assuming $ 360 $ days and $ 2880 $ steps per year.

%Additional matrices for coarser temporal resolutions can be provided (see \cite{Kha07}).

%
%	Data Hierarchy
%
\subsection{Data Hierarchy}

The data is organized in a hierarchy of directories starting with the
root directory named \texttt{Metos3DData}.
The \textbf{main distinction} between data files is the \textbf{spatial resolution}:
\begin{verbatim}
.../Metos3DData/
    2.8/
\end{verbatim}

For a spatial resolution you will find a distinction
into \textbf{geometry}, \textbf{initialization}, \textbf{forcing} and
\textbf{transport} within that directory:
\begin{verbatim}
.../Metos3DData/2.8/
    Geometry/
    Initialization/
    Forcing/
    Transport/
\end{verbatim}

The geometry and initialization directories have no further subdirectories.
The forcing is divided into \textbf{boundary} and \textbf{domain conditions}.
\begin{verbatim}
.../Metos3DData/2.8/Forcing/
    BoundaryCondition/
    DomainCondition/
\end{verbatim}

The transport matrices are organized within the transport directory.
The first level of subdirectories denote the \textbf{kind} of matrices (see \cite{KhViCa05}).
The second distinguishes between \textbf{temporal resolutions}.
%The \texttt{1dt}
%directory corresponds to the $ 3 $ hour time step, \texttt{2dt} to the $ 6 $
%hour time step and so on:
\begin{verbatim}
.../Metos3DData/2.8/Transport/
    Matrix5_4/
        1dt/
\end{verbatim}

%
%	References	%%%%%%%%%%%%%%%%%%%%%%%%%%%%%%%%%%%%%%%%%%%%%%%%
%
\bibliographystyle{plain}
%\bibliography{/Users/jpicau/Documents/ARBEIT/CODE/Literature/literature}
\begin{thebibliography}{1}

\bibitem{KhViCa05}
S.~Khatiwala, M.~Visbeck, and M.A. Cane.
\newblock Accelerated simulation of passive tracers in ocean circulation
  models.
\newblock {\em Ocean Modelling}, 9(1):51--69, 2005.

\end{thebibliography}


\end{document}
